\documentclass[requirements.tex]{subfiles}
\begin{document}
\section{Target Users \& Main Tasks} % (fold)
\label{sec:target_users_main_tasks}

\subsection{Stakeholder Analysis} % (fold)
\label{sub:stakeholder_analysis}
Since the scope of this project revolves around a smaller, specific part of a
larger system of interaction, the most simplistic view of the stakeholder of the
new text entry system would be `people who use video game consoles'. While a
common assumption would be that this would would mean that a text entry should
primarily focus on people aged 24 and under, studies have shown that this is
not the case. 

A 2015 study by Pew Research Center \cite{pew} claimed that in the US, 40\% of
adults owned a console such, with 56\% of those aged 18-29 and 55\% of those
aged 30-49 saying they do. This means that the system must cater to a wide
userbase in terms of age. The same is true for gender, as the study also claimed
that 37\% of men and 42\% of women own consoles as well.

The Electronic Software Association, a trade association and lobbyist group
whose members include major publishers and console manufacturers also released a
study \cite{esa} on demographics that showed both age and gender demographics.
This study showed more males (56\%) than females (44\%) while the age breakdown
was 26\% for 18 and under and 30\% for 18-35.

This data shows that there are a wide variety of large groups of users who use
consoles, whether they be children or adults with at least some disposable
income. Since text entry on games consoles is usually operating-system level
functionality (rather than each game having its own), this means that a text
entry system should be understood by all of these demographics. This is likely
why the slower yet simpler current standard exists, with most sophisticated
systems being optional.
% subsection stakeholder_analysis (end)

\subsection{User Persona} % (fold)
\label{sub:user_persona}
Based on the data provided in \ref{sub:stakeholder_analysis}, I believe that my
proposed text entry system will be more focused on adults aged between 18-35 who
own video game consoles as the main focus of my system.

These are the users who most likely to easily adapt to the new system due to
their age, being old enough to handle complexity but young enough to still have
a major interest in video games. In addition, these users are likely to have
some disposable income, meaning they are more likely to use multimedia
applications such as Netflix or browsing online stores to purchase games. This
increases the number of instances in which the user will be required to perform
text entry. They are the users who will want to send messages to friends to.
organise group sessions as well. In addition, I believe that these users would
be more interested in changes made in usability and quality of live improvements
to these systems as well.
% subsection user_persona (end)


% section target_users_main_tasks (end)
\newpage
\end{document}