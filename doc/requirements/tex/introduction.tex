\documentclass[requirements.tex]{subfiles}
\begin{document}
\section{Introduction} % (fold)
\label{sec:introduction}

The console video game market is a highly lucrative industry to be a part of as
both a software and hardware developer. In May of 2016, Sony announced that
sales of their Playstation 4 console had surpassed 40 million units, with over
270 million software units sold in the same period, since the launch of the
system in November 2013\cite{sony}. When combined with Microsoft's Xbox One
platform and Nintendo's WiiU (and upcoming Switch) consoles, the home console
industry is one with a major presence today.

All of these consoles require text input as part of their functionality. Whether
it be naming a character in a role-playing game, searching for a product on the
system's online store or multimedia applications, or simply just sending a
message to a friend, there are various use cases in which the user is required
to input text using the console's controller.

However, text entry using a game controller is still a developing area of
research and development, with more sophisticated methods only being
investigated relatively recently. Until online functionality became a
cornerstone in the seventh generation of home consoles with the XBox 360 and
Playstation 3, text input with a controller was much more limited, with games
requiring characters to be named being the only use case most players would
run into. This meant that the early days of text input was very limited, usually
requiring the user to scroll one character at a time in order to type the name
or phrase required.

The seventh generation saw the consoles implement their own on-screen keyboards
at the OS level for all games to use, meaning that a uniform standard was then
found on each system. This on-screen keyboard would mimic a traditional Qwerty
keyboard, but would still only allow the user to traverse via access of a single
character. This means that typing on a games console can be much slower compared
to typing on a regular keyboard, which is problematic when certain use cases may
require longer phrases to be typed in, worsening the user experience.

The aim of this research is to devise a new alternative approach to text input
utilising a game controller. The design will revolve around allowing the user
to have access to more characters at once in order to improve the speed of
typing while also not requiring a steep learning curve or deviation from the
Qwerty layout that is already familiar with the users. This topic was of
interest to me due to two main aspects: text entry in this context is a task
which is constrained by the input device requires, and it is a smaller task
that is repeated very frequently, resulting in an outcome that is somewhat death
by a thousand cuts. I believe that a text entry design that improved typing
speed without increasing error rate would have strong long term results in the
time taken to perform this task.

\subsection{Report Structure} % (fold)
\label{sub:report_structure}

Section \ref{sec:literature_review} will outline other studies into this same
topic, describing their approaches, and how they have influenced the design of
this system.
This is followed by Section \ref{sec:market_analysis}, looking at alternatives
to text input that are already available to use on consoles (as well as on PC
gaming) platforms today.
Section \ref{sec:target_users_main_tasks} will outline the stakeholder analysis
and use cases of the proposed solution, with Section \ref{sec:requirements}
outlining the requirements of the system, both functional and non-functional.
Finally, Section \ref{sec:initial_designs} will include the initial designs and
prototypes of the proposed solution.
% subsection report_structure (end)

% section introduction (end)
\newpage
\end{document}