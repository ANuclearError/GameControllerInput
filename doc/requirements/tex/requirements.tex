\documentclass[requirements.tex]{subfiles}
\begin{document}
\section{Requirements} % (fold)
\label{sec:requirements}

\subsection{Input} % (fold)
\label{sub:input}
The user must be able to use the left analogue stick of the controller to move
the cursor matrix on the display. This indicates which group of characters the
user wishes to select from. The user must also be able to use the right stick to
then select which specific character they wish to input, with the direction the
stick is pushed selecting the character. Pressing down on the right stick must
select the middle character of the matrix.

The user must also be able to use the trigger buttons to move the caret either
left or right depending on which is pressed. The user must also be able to
backspace or add a space using the bumpers in a similar way.

The user must also be able to switch between upper and lower case letters (or
punctuation marks). The user must be able to submit the text by pressing the
start button. The user should be able to cancel the text input by pressing a
button, although for prototype purposes, this will simply clear text instead.
% subsection input (end)

\subsection{Character Sets} % (fold)
\label{sub:character_sets}
The system must allow the user to input alphanumeric characters, both upper and
lower case, with basic punctuation marks to round the resulting matrix to 40
characters (10 numbers, 26 letters and 4 punctuation marks). The system should
also support the remaining punctuation marks. The system could allow for accents
to be applied to certain characters as well, while it would also support
additional languages such as Greek or Cyrillic.
% subsection character_sets (end)

\subsection{Display} % (fold)
\label{sub:display}
The on-screen keyboard must display a matrix of characters that acts as the
keyboard. It must also highlight the part of this matrix that corresponds to
where the cursor is currently positioned. It must also show the user the current
text they have input, as well as the position of the caret indicating where text
is being input. The display should respond quickly to any input from the user,
in order to prevent the user from needlessly waiting. The display could also
display predictive text, although this feature is already provided by consoles,
and would not need changed to integrate with this new system.
% subsection display (end)
% section requirements (end)
\end{document}