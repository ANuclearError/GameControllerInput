\documentclass[future.tex]{subfiles}
\begin{document}
\section{Future Work and Recommendations} % (fold)
\label{sec:future_work_and_recommendations}
The current stage of the prototype allows for many opportunities with regards
to future work that can, with many ideas that could not be investigated without
a more long term strategy in mind. Along with improvements to the current
prototype (taking both the 2x2 and 3x3 versions forward at this stage), there
are more extreme variations that stick to the core `select a character from a
grid using the right control tick' feature that was the primary focus of the
project. In addition, there is a basis for future evaluation of the designed
system as it is refined and iterated upon.

\subsection{Future Improvements} % (fold)
\label{sub:future_improvements}
The immediate improvements that should be made is a mastering of how the right
control stick movement corresponds to selected characters in the 3x3 version of
the prototype. Dividing the joystick movement into 8 sections of 45 degrees each
proved to be a source of difficulty to users, and thus there needs to be a way
to improve upon this. While it is likely that users would become used to the
movement with extended usage, it would be ideal to ensure that the experience
for novice users of the prototype do not have these issues.

After finishing tweaking of the selection system, the future work with the
prototype would be expanding the features of text input method to match the
expected features of an on-screen keyboard. This includes the features such as
changing cases, allowing for more punctuation marks and moving the caret to the
left and right in the input text. While the inclusion of these features was
deemed superfluous to evaluating the character selection, it would be beneficial
to include these features at this stage as well, so that the next stages would
be performed with all features being available.
% subsection future_improvements (end)

\subsection{Other Approaches} % (fold)
\label{sub:other_approaches}
There are more alternative designs that could be considered alongside the one
proposed, which branch off the prototype designed as part of this project.
After taking the 2x2 grid version of the prototype on board after user testing,
it was realised that it could potentially be combined with the Steam controller
entry method outlined in Section \ref{sub:steam_controller}.

The basic concept of this design would involve splitting the keyboard into
two sections vertically and having two separate 2x2 cursor grids rather than
just the one in the current prototype. These cursors would have two modes: move
mode (the default) and selection mode. Selection mode would be triggered by the
holding of a button (the left and right bumpers being the most appropriate for
each cursor), at which point movement of the control stick would select one of
the four characters rather than move the cursor. Selection of a character could
be confirmed by the release of the button, or instead the user could select
multiple characters while the button is held. The prototype would require some
modification, mainly to handling events triggered by user actions with the
controller. However, the prototype should be able to handle multiple cursors
without any issues.

If considering this alternative design, then the primary concern during
evaluation would be the user's co-ordination while controlling two cursors.
Some users may struggle with using both cursors simultaneously, resulting in
reduced efficiency. However, these problems are ones that Valve have already had
to consider when designing their controller, meaning they must believe that this
does not present a problem for the majority of users.
% subsection other_approaches (end)

\subsection{Future Evaluation} % (fold)
\label{sub:future_evaluation}
The primary concern regarding evaluation during any future work with the design,
is the need for users to be able to use the text entry method over a much longer
period of time. All the studies considered in the literate review (Section
\ref{sec:literature_review}) show that a long term approach to evaluation was
taken with the implementation of their designs.

A long term evaluation would involve a larger group of users than those used
during the user testing and analytic evaluation stages of this design process.
After gaining baseline comparisons with the standard system from the users,
users would then use the proposed design over an extended period of time, at
least two weeks long. The evaluation would focus on the rate of improvement as
the user becomes familiar with the input controls, providing a more accurate
picture of the suitability of the chosen design. In addition, the users would
be typing a wider variety of phrases and paragraphs as well, using a mixture of
upper cases, lower cases and punctuation marks to ensure that the widest range
of possibilities are considered as part of the testing.

This form of evaluation would ensure that the users have the opportunity to
become familiar and comfortable with these more complex text entry methods, and
show whether or not this entry method provides the long term improvements to
typing speed and error prevention do exist for the users.
% subsection future_evaluation (end)

Overall, the design and prototyping of this potential solution to game
controller text input provides a wide berth of directions to move forward in.
The core concept of the grid cursor for selecting characters is flexible enough
that means that iteration upon it can be performed while keeping to this core.
Considering the improvements already made from users' first experiences with the
prototype, then it does suggest that extended exposure with the system for users
would provide more feedback that can improve the input. As such, it could then
be determined how well this design solves the original problem of text entry
with controllers.
% section future_work_and_recommendations (end)
\end{document}