\documentclass[testing.tex]{subfiles}
\begin{document}
\section{User Testing} % (fold)
\label{sec:user_testing}
User testing was performed throughout the design process, using not just the
low fidelity prototype outlined in Section \ref{sec:initial_designs}, but also
increasingly higher fidelity prototypes as they were developed. A basic
presentation was created to go through the various user actions, with a game
controller being used for visual reference as well. The user testing was done
with half a dozen users, all of whom were aged 20-24.

The initial user testing took the form of a walkthrough of the low fidelity
prototype, with the controller being used to demonstrate the actions. Users
were also asked questions with regards to their experiences with this task as
well. The user testing was not completely structured, with feedback being
recorded as notes. This was done to allow for flexibility when discussing with
the individual users, since there were a variety of experience levels with the
use of game controllers, it would have been difficult to come up with a survey
to handle all possibilities. The key takeaways from the user testing were as
follows:

\begin{itemize}
	\item Visual feedback of errors should be provided as well as where the
    carat is must be provided.
	\item The movement of the cursor must use the directional pad rather than
	the left control stick.
	\item The selection grid should be 2x2 rather than 3x3.
	\item The selection with right stick could have an addition button press for
	confirmation.
	\item The movement and selection should have balanced sensitivity, to
    prevent characters or movements being repeated when not desired.
\end{itemize}

The first three requirements were easily accommodated into the design, with a
decision being made to create high fidelity prototypes using both the 2x2 and
3x3 version to determine which was more suitable in regards to typing speed and
user comfort. The last requirement users suggested was due to their concerns
with the preciseness of the control stick when being used for 8 directions. It
was decided that this requirement was not added to the high fidelity prototype
since it was believed that the implementation would not require this precaution.
In hindsight, it perhaps should have been included at least once, in order to 
see what difference it makes to the end results of evaluation. This can be
considered for future work (Section \ref{sec:future_work_and_recommendations}).

Overall, the users were interested in the direction being taken with this
proposed design. Much of the given feedback was helpful, such as the suggestion
of trying a 2x2 version as well, in addition to the extra visual feedback
desired. The concerns raised about the sensitivity did enforce the initial
concerns with the high fidelity prototype, since the choice of language and
library did provide the possibility of pitfalls being encountered that could
impact users.
% section user_testing (end)
\end{document}