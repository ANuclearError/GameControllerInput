\documentclass[evaluation.tex]{subfiles}
\begin{document}
\section{Analytic Evaluation} % (fold)
\label{sec:analytic_evaluation}
\subsection{Evaluation Process} % (fold)
\label{sub:evaluation_process}
As discussed in the literature review (Section \ref{sec:literature_review}), the
two primary metrics to be considered when evaluating the prototype created are:
\begin{itemize}
	\item The user's words per minutes (WPM) while typing.
	\item The error rate of the user while typing.
\end{itemize}

A simple test was created to examine how users handled the new text entry method
on their first attempts. While ideally, this would be done over an extended
period of time rather than being a short term evaluation, time did not permit
such possibilities. The users were given the phrase `the quick brown fox jumped
over the lazy dog' to type using three text entry methods.
\begin{itemize}
	\item The `standard' entry method found in existing game consoles, acting as
	the baseline for comparison.
	\item The `2x2' entry method created after user testing, where the user
	selects from a 2x2 matrix of characters by moving the right control stick.
	\item The `3x3' method originally proposed, with the right stick selecting
	from a 3x3 matrix of characters (utilising both movement and pressing of
	the stick).
\end{itemize}

This test was performed twice with the six users who had volunteered, all of
whom had some technical expertise and experience with controllers. The first
test was done after implementation of the prototype was completed, allowing for
an opportunity to tweak the prototype in minor ways to reflect user results in
addition extra feedback given from the users. Again, users were not given a
set survey to complete after their testing, instead again opting for a more
loose conversation instead.

The user testing combined with the initial result below allows for modification
of the prototype for another round of tests. User issues were mainly focused on
the sensitivity of the controller, which would treat small movements as actions
that the users did not always want. In addition, the users expressed desire for
the directional pad to control movement instead of the left joystick. 

Overall, the evaluating of this evaluating felt like an extension to the user
testing stage of the design process (Section \ref{sec:user_testing}), being
somewhat of a merging of the two stages. The initial prototype results can thus
arguably be considered as a part of user testing also, as well as part of this
evaluation.
% subsection evaluation_process (end)

\subsection{Results} % (fold)
\label{sub:results}
\subsubsection{Initial Prototype} % (fold)
\label{ssub:initial_prototype}
\begin{table}[H]
	\centering
	\begin{tabular}{| l | l | l | l | l | l | l |}
	 \hline
	 \textbf{User} & \multicolumn{2}{c|}{\textbf{Standard}} &
	 \multicolumn{2}{c|}{\textbf{2x2}} & \multicolumn{2}{c|}{\textbf{3x3}} \\
	 \cline{2-3} \cline{4-5} \cline{6-7}
	  & Time (s) & Errors & Time (s) & Errors & Time (s) & Errors\\
	 \hline
	 1 & 69.215 & 3 &  &  & 70.191 & 7 \\
	 2 & 83.228 & 5 & 81.089 & 1 & 96.185 & 11 \\
	 3 & 144.9 & 1 & 173.039 & 13 & 170.012 & 33 \\
	 4 & 73.592 & 3 & 73.024 & 1 & 108.111 & 20 \\
	 5 & 125.988 & 7 & 144.149 & 6 & 129.349 & 11 \\
	 6 & 94.494 & 4 & 94.42 & 4 & 132.525 & 15 \\
	 \hline
	 Average & 98.57 & 3.83 & 105.88 & 4.83 & 117.73 & 16.17 \\
	 \hline
	\end{tabular}
	\caption{The results obtained from the initial prototype after basic
	user testing.}
\end{table}
% subsubsection initial_prototype (end)

\subsubsection{Modified Prototype} % (fold)
\label{ssub:modified_prototype}
\begin{table}[H]
	\centering
	\begin{tabular}{| l | l | l | l | l | l | l |}
	 \hline
	 \textbf{User} & \multicolumn{2}{c|}{\textbf{Standard}} &
	 \multicolumn{2}{c|}{\textbf{2x2}} & \multicolumn{2}{c|}{\textbf{3x3}} \\
	 \cline{2-3} \cline{4-5} \cline{6-7}
	 & Time (s) & Errors & Time (s) & Errors & Time (s) & Errors \\
	 \hline
	 1 & 48.226 & 1 & 61.123 & 3 & 89.552 & 17 \\
	 2 & 62.779 & 0 & 69.419 & 0 & 90.056 & 13 \\
	 3 & 64.8 & 1 & 71.435 & 2 & 88.442 & 7 \\
	 4 & 52.342 & 0 & 55.225 & 1 & 69.937 & 4 \\
	 5 & 65.815 & 5 & 62.724 & 9 & 74.874 & 20 \\
	 6 & 61.819 & 0 & 70.57 & 0 & 81.162 & 8 \\
	 \hline
	 Average & 59.30 & 1.17 & 65.08 & 2.50 & 82.34 & 11.50 \\
	 \hline
	\end{tabular}
	\caption{The results obtained from the prototype modified after user
	testing and previous test.}
\end{table}
% subsubsection modified_prototype (end)
% subsection results (end)

\subsection{Analysis} % (fold)
\label{sub:analysis}
Overall, the results show that for user's initial experiences with the
prototypes does show that a learning curve exists with the increasingly complex
entry methods.  Users were generally more comfortable with the 2x2 method rather
than the 3x3 method, citing continuing issues with the 8-way joystick movement
as the main source. Since the test was designed to be a phrase making maximum
use of the keyboard, many users struggled with adding the full stop punctuation
to the end of the phrase, since it was in the bottom-right corner of the
keyboard. 

It was interesting to note that through all input methods, some users needed
time to recalibrate themselves when it came to moving the cursor, sometimes
forgetting where the key they desire is, despite keeping to the QWERTY layout
which should already be familiar to the users.

Users felt that the 2x2 was a better implementation of the ideas brought
forward by this design process than the 3x3 method that was the initial concept
at the start of design. However, there was some consensus that this was not
automatically an issue with the concept itself, but potentially just a problem
with implementation. Further tweaking and experimenting with a variety of
controllers would be required to confirm this.

Overall, it is clear that users cannot immediately transition from the
traditional input method to the styles proposed by this design process. However,
it is the belief of the designers, based on their own extended use of the
prototype during development, that users will see improvements to the times
obtained during evaluation if given the ability to use the system for a longer
period of time. Further work should also be done in order to perfect the 8-way
controller movement, which is the primary issue for users. However, these
issues did not exist for the 2x2 method, which required only 4 directions of
movement instead, which made typing easier.
% subsection analysis (end)
% section analytic_evaluation (end)
\end{document}