\documentclass[conclusions.tex]{subfiles}
\begin{document}
\section{Conclusions} % (fold)
\label{sec:conclusions}
The design approach taken for the design of this new controller text entry
method saw an incremental build week by week, that correspond to the sections
of this report. It was advantageous to have this structure as a roadmap, since
it showed a clear progression from ideas in the head, to rough low fidelity
prototypes to finally the high fidelity prototype. It was also a useful
approach since it ensured that users were consulted early and frequently, rather
than waiting right until the end for their involvement (and avoiding the
subsequent issues that would result in). However, when going through the design
approach, there were some instances where flexibility was required, such as
with the user testing and analytic evaluation, which ended up having more
blurred lines between them.

The literature review stage of the design proved to be a valuable part of the
design process, since the research undertaken at this stage showed the variety
of entry methods that had been studied previously. In addition, it helped to
provide examples to how to handle other aspects of the design, such as the end
evaluations. The market analysis provided similar inspirations, due to the
variety shown in both the literature and market that has largely been ignored
by the mass market.

The target user and task analysis was less useful by comparison however, since
the users of games consoles are a diverse group and the fact that the tasks
being performed are rather straightforward, being text entry.

The requirements gathering was somewhat useful, with the main advantage being
that it helped to prioritise what should be implemented as part of the high
fidelity prototype, as well as allowing for contributions from the user as well.
The user testing also helped in this respect, although it did end up being done
alongside the high fidelity prototype and analytic evaluation in addition to
testing with the low fidelity prototype. This is where flexibility was needed to
allow for the user suggestions at all stage to be added.

The analytic evaluation was mostly useful in seeing user's initial experience
with the system, although this is mainly due to time management. It is clear
that more time being dedicated to this stage would have allowed for more
interesting and useful evaluation, since the problem being tackled by the
system is one that needs considered from a long term approach.

Overall, the design approach undertaken was beneficial to the improvement of
the system and the resulting design of it. Without the structure in place, there
would have been a desire to jump straight to making the high fidelity prototype
where users become concerned with granular issues such as exact joystick
sensitivity. The early stages of the design approach meant that the prototype
was able to be more flexible to allow for an additional system to be looked at
in addition to the original concept from the design. In addition, there were
various other ideas that were formed, looking at different alternatives to the
text entry input proposed here (see Section
\ref{sec:future_work_and_recommendations}) that could be considered in the
future.

While the users could not immediately improve their typing speed and accuracy
with the prototypes, there is still potential in the designs that could be
unlocked with continuous use of the system instead of a singular use of the
system. The 2x2 design that was considered after user discussion shows that
there is a demand for further work in this area, and that this design process
has resulted in a strong start to improving the original problem scope outlined
at the start of design.
% section conclusions (end)
\end{document}