\documentclass[requirements.tex]{subfiles}
\begin{document}
\section{Market Analysis} % (fold)
\label{sec:market_analysis}

There are some examples in the market of different approaches to text input
using game controllers, although they are only on two different platforms:
Sony's Playstation 4 console and Valve's Steam platform on PC. A casual overview
from popular video game website Kotaku \cite{Kotaku} provides visual support,
showing videos and GIFs of these systems being performed.

\subsection{Dualshock 4 Gyroscopic} % (fold)
\label{sub:dualshock_4_gyroscopic}
The Playstation 4's Dualshock 4 controller has gyroscopic technology included,
which allows for a input method utilising it. The keyboard still has the single
character selection of the traditional method, but now its movement is
controlled by the movement of the physical controller rather than button presses
or analogue stick movement. This does result in faster movement from one side of
the keyboard to another, although there is the draw back of this being a much
more physical activity which can become tedious on longer usage. Another method
of manipulating the cursor during text input is using the controller's touch pad
instead of using the directional pad or analogue stick, although this method
does not feel different to the user in practise in comparison, making it not as
interesting as an example.

\begin{figure}[H]
	\centering
	\includegraphics[width=0.75\textwidth]{img/ps4-gyro}
	\caption{Gyroscopic text input on the Playstation 4. The circle represents
	the controller's movement, while the rectangular outline represents the
	character being chosen}
\end{figure}
% subsection dualshock_4_gyroscopic (end)

\subsection{Steam Controller} % (fold)
\label{sub:steam_controller}
Valve's offerings in this subject are are more interesting, with it being clear
that this is a challenge that they have approached with much consideration.

The first solution is based on their own controller, the Steam Controller, which
is a games controller that provides two trackpads and one analogue stick, rather
than sticking to the traditional controller layout of two analogue sticks and
direction pad. This means that they had an opportunity to provide a new way of
text input. The text entry system found is can be considered a modification of
Wilson \& Agrawala's entry system (Section \ref{sub:dual_stick_methods}),
modified to use the trackpads rather than sticks.

The two trackpads control two separate cursors instead of just the one, with
shoulder buttons being used to enter characters. This method offers the
advantage of it being much quicker to enter character sequences such as `qu',
which is appreciated. Wilson \& Agrawala's study showed improved performance
with this split keyboard, although it should be considered that this system does
require the Steam controller, and is not available to other controllers that can
be used on the platform.

\begin{figure}[H]
	\centering
	\includegraphics[width=0.6\textwidth]{img/steam}
	\caption{Steam Controller on-screen keyboard. The two cursors are
	controlled by the left and right trackpads.}
\end{figure}
% subsection steam_controller (end)

\subsection{Steam Big Picture Mode} % (fold)
\label{sub:steam_big_picture_mode}
The second such solution is found in their `Big Picture' mode, which provides an
interface designed for being use at a longer distance between user and display.
The keyboard is divided into 8 groups of 4 characters, arranged in a circle.
The left analogue stick chooses between each of these groups, depending on the
position of the stick. The face buttons (A, B, X, Y on XBox or Cross, Circle,
Square, Triangle on Playstation) are then used to choose the specific character.

Out of the three solutions found in market, it is this approach from Valve that
is the most interesting, due to its ability to tackle the goal of allowing
more character access at once. However, the key arrangement demanded by this
input method is too different from the traditional method, which results in a
learning curve that may be too large for the task at hand. This proposed design
for text input is an attempt to take the mindset shown by Valve, but applied to
take advantage of the traditional keyboard layout.

\begin{figure}[H]
	\centering
	\includegraphics[width=0.55\textwidth]{img/steam-bp}
	\caption{Steam Big Picture mode's on-screen keyboard.}
\end{figure}
% subsection steam_big_picture_mode (end)
% section market_analysis (end)
\end{document}