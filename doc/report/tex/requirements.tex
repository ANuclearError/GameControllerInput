\documentclass[requirements.tex]{subfiles}
\begin{document}
\section{Requirements} % (fold)
\label{sec:requirements}
The requirements were gathered and divided using the MoSCoW method, resulting
in four sections: `must have', `should have', `could have' and `won't die'. The
requirements tables show the before and after affect of the user testing.

\subsection{Before User Testing} % (fold)
\label{sub:before_user_testing}
\subsubsection{Must Have} % (fold)
\label{ssub:before_must_have}
\begin{table}[H]
\small
\begin{tabularx}{\textwidth}{| l | X | l |}
 \hline
 \textbf{Requirement} & \textbf{Description} & \textbf{Functional} \\
 \hline
 Keyboard display & The on-screen keyboard must display a grid of selectable
 characters, acting as the keyboard. & Functional \\
 \hline
 Cursor display & The area of characters that the user is able to select from
 must be highlighted to the user, to allow them to see the position of the
 cursor in the keyboard. & Functional \\
 \hline
 Input Display & The display must show the text the user has already input, as
 well as the carat showing text is being input. & Functional \\
 \hline
 Alphabet & The keyboard must also for the selection of alphanumeric characters,
 with additional punctuation marks to round the total number available. &
 Functional \\
 \hline
 Spacing & The user must have a way to add a space to the text they are
 entering. & Functional \\
 \hline
 Movement & The user must be able to move the cursor, using the left control
 stick. & Functional \\
 \hline
 Backspacing & The user must have a way to backspace when they wish to change
 the text. & Functional \\
 \hline
 Selection & The user must be able to select from the grid of characters, using
 the right control stick. Where the user moves the stick determines which
 character is selected. & Functional \\
 \hline
 Move carat & The user must be able to use the triggers to move the carat left
 or right, changing where text is being input. & Functional \\
 \hline
 Switch case & The user must be able to switch between upper and lower case.
 & Functional \\
 \hline
 Submission & The user must be able to submit their finished input using the
 options button. & Functional \\
 \hline
\end{tabularx}
\end{table}
% subsubsection before_must_have (end)

\subsubsection{Should Have} % (fold)
\label{ssub:before_should_have}
\begin{table}[H]
\small
\begin{tabularx}{\textwidth}{| l | X | l |}
 \hline
 \textbf{Requirement} & \textbf{Description} & \textbf{Functional} \\
 \hline
 Punctuation & The keyboard should allow for switch to a wider selection of
 punctuation marks. & Functional \\
 \hline
 Responsiveness & The display should quickly update to match the user's input,
 so that fast sequences of typing are accurate. & Non-functional \\
 \hline
\end{tabularx}
\end{table}
% subsubsection before_should_have (end)

\subsubsection{Could Have} % (fold)
\label{ssub:before_could_have}
\begin{table}[H]
\small
\begin{tabularx}{\textwidth}{| l | X | l |}
 \hline
 \textbf{Requirement} & \textbf{Description} & \textbf{Functional} \\
 \hline
 Predictive Text & The display could suggest words to the user, providing a
 mechanism for auto-completion & Functional \\
 \hline
 Accents & The keyboard could include a mechanism for adding accents to
 characters in the Latin Alphabet, for languages such as French. & Functional \\
 \hline
\end{tabularx}
\end{table}
% subsubsection before_could_have (end)

\subsubsection{Would Have} % (fold)
\label{ssub:before_would_have}
\begin{table}[H]
\small
\begin{tabularx}{\textwidth}{| l | X | l |}
 \hline
 \textbf{Requirement} & \textbf{Description} & \textbf{Functional} \\
 \hline
 Different alphabets & The system would be able to handle different alphabets
 such as Greek or Cyrillic, with the display shape being altered to match this.
 & Functional \\
 \hline
\end{tabularx}
\end{table}
% subsubsection before_would_have (end)
% subsection before_user_testing (end)
\subsection{After Using Testing} % (fold)
The following tables outline the changes made to requirements after user
testing. \emph{Emphasised} text indicates a new or altered requirement, while
\textbf{bold} text indicates that this requirement was implemented in the high
fidelity prototype. If the proposed keyboard system were to be fully developed
for a games console, then all the `must have' requirements would be necessary,
but since the prototype was focused on the character method aspect of the
input, it was felt that a particular focus was required for the prototype.

In addition to the user contributions, an expansion was made with regards to
non-functional requirements.

\subsubsection{Must Have} % (fold)
\label{ssub:after_must_have}
\begin{table}[H]
\small
\begin{tabularx}{\textwidth}{| l | X | l |}
 \hline
 \textbf{Requirement} & \textbf{Description} & \textbf{Functional} \\
 \hline
 \textbf{Keyboard display} & The on-screen keyboard must display a grid of
 selectable characters, acting as the keyboard. & Functional \\
 \hline
 \textbf{Cursor display} & The area of characters that the user is able to
 select from must be highlighted to the user, to allow them to see the position
 of the cursor in the keyboard. & Functional \\
 \hline
 \textbf{Input Display} & The display must show the text the user has already
 input, as well as the carat showing text is being input. & Functional \\
 \hline
 \textbf{Alphabet} & The keyboard must also for the selection of alphanumeric
 characters, with additional punctuation marks to round the total number
 available. & Functional \\
 \hline
 \textbf{Spacing} & The user must have a way to add a space to the text they are
 entering. & Functional \\
 \hline
 \textbf{\emph{Movement}} & The user must be able to move the cursor, using the
 directional pad. & Functional \\
 \hline
 \textbf{Backspacing} & The user must have a way to backspace when they wish to
 change the text. & Functional \\
 \hline
 \textbf{Selection} & The user must be able to select from the grid of
 characters, using the right control stick. Where the user moves the stick
 determines which character is selected. & Functional \\
 \hline
 Move carat & The user must be able to use the triggers to move the carat left
 or right, changing where text is being input. & Functional \\
 \hline
 Switch case & The user must be able to switch between upper and lower case.
 & Functional \\
 \hline
 \textbf{Submission} & The user must be able to submit their finished input
 using the options button. & Functional \\
 \hline
\end{tabularx}
\end{table}
% subsubsection after_must_have (end)

\subsubsection{Should Have} % (fold)
\label{ssub:after_should_have}
\begin{table}[H]
\small
\begin{tabularx}{\textwidth}{| l | X | l |}
 \hline
 \textbf{Requirement} & \textbf{Description} & \textbf{Functional} \\
 \hline
 Punctuation & The keyboard should allow for switch to a wider selection of
 punctuation marks. & Functional \\
 \hline
 Responsiveness & The display should quickly update to match the user's input,
 so that fast sequences of typing are accurate. & Non-functional \\
 \hline
 \emph{Error Feedback} & The word currently being typed should have visual
 feedback if it is misspelt. & Functional \\
 \hline
 \textbf{\emph{Smaller Matrix}} & The selection matrix should be 2x2 rather than
 3x3, due to concerns that 8 way movement with the control stick could result
 in increased errors. & Functional \\
 \hline
 \emph{Resources} & The system should have minimal use of resources, requiring
 little memory or processing time. & Non-functional \\
 \hline
 \emph{Maintainability} & The keyboard design should be maintainable to allow
 for easier modification, to allow for example, a larger grid or different
 alphabets. & Non-functional \\
 \hline
 \emph{Reliability} & The movement of the right control stick for character
 selection should be reliable so that the user's desired character matches their
 movement, without being askew. & Non-functional \\
 \hline
 \emph{Data Integrity} & The system should ensure that the input being provided
 by the user is correct, and remains the same unless the user has specifically
 performed an action. & Non-functional \\
 \hline
\end{tabularx}
\end{table}
% subsubsection after_should_have (end)

\subsubsection{Could Have} % (fold)
\label{ssub:after_could_have}
\begin{table}[H]
\small
\begin{tabularx}{\textwidth}{| l | X | l |}
 \hline
 \textbf{Requirement} & \textbf{Description} & \textbf{Functional} \\
 \hline
 Predictive Text & The display could suggest words to the user, providing a
 mechanism for auto-completion & Functional \\
 \hline
 Accents & The keyboard could include a mechanism for adding accents to
 characters in the Latin Alphabet, for languages such as French. & Functional \\
 \hline
 \emph{Character confirmation} & When the user wishes to select a character, a
 button press could be required in addition to control stick movement to ensure
 accuracy. & Functional \\
 \hline
\end{tabularx}
\end{table}
% subsubsection after_could_have (end)

\subsubsection{Would Have} % (fold)
\label{ssub:after_would_have}
\begin{table}[H]
\small
\begin{tabularx}{\textwidth}{| l | X | l |}
 \hline
 \textbf{Requirement} & \textbf{Description} & \textbf{Functional} \\
 \hline
 Different alphabets & The system would be able to handle different alphabets
 such as Greek or Cyrillic, with the display shape being altered to match this.
 & Functional \\
 \hline
\end{tabularx}
\end{table}
% subsubsection after_would_have (end)
% subsection after_using_testing (end)
% section requirements (end)
\end{document}