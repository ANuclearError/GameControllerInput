\documentclass[prototype.tex]{subfiles}
\begin{document}
\section{High Fidelity Prototype} % (fold)
\label{sec:high_fidelity_prototype}
\subsection{Prototype Format} % (fold)
\label{sub:prototype_format}
The high fidelity prototype became a basic implementation of the system being
proposed, interactive with a game controller and including some performance
measuring abilities. Due to how the nature of the topic necessitates the use of
a game controller when talking to users about the design, as well as the
somewhat simpler nature of the tasks being performed, the decision to work
towards a more complete implementation felt reasonable with the deadlines
imposed. However, it did have the downside that user feedback did relate to the
granular aspects of the implementation, to some detriment to evaluating the
design concept as a whole.
% subsection prototype_format (end)

\subsection{Development Environment} % (fold)
\label{sub:development_environment}
The prototype was developed in C, with heavy use of the Simple DirectMedia Layer
library. SDL is a library that handles low-level access to hardware devices such
as keyboard or game controller events, as well as handling graphics rendering as
well. This meant that the rendering of the keyboard, the game controller event
handling, and other aspects of the system were handled by a single library,
making the development of the prototype easier.

SDL was chosen due to it being a popular library in video game
development, being used in various popular games and engines such as Valve's
Source Engine. This meant that there were many resources available for learning
the library, as well as sufficient documentation provided to allow for easier
development with the library.
% subsection development_environment (end)

\subsection{Prototype Features} % (fold)
\label{sub:prototype_features}
The prototype included a single 40 character keyboard, including the lower case
alphabet, digits and some punctuation marks. The prototype allowed for the user
to perform text entry using a game controller, controlling the movement of a
square grid (between 1 and 3 keys in size) and the selection of characters
within this grid. The prototype also included spacing and backspacing as well,
since these were necessary to error handling and in completing full sentences
for the evaluation.

Since the focus of the project was specifically about the character selection,
shifting to upper case characters or more punctuation marks was omitted, since
these actions would not change with the proposed design. In addition, the
ability to move the carat left to right in the input was not included, since
they were again not being included, and it was felt that the user would not
need them for evaluation purposes. The prototype was used for fair comparison
between the standard method, and the designed ones.
% subsection prototype_features (end)

\subsection{Development Lifecycle} % (fold)
\label{sub:development_lifecycle}
Before developing the prototype, it was required to implement the basic input
method currently used in consoles. This was required so that comparisons could
be made between the prototype and the original system. The prototype's code was
designed and implemented in a way to allow for the size of selection matrix to
be modified to 2x2 or 3x3 grids, which were the two prototypes to be evaluated.
% subsection development_lifecycle (end)

\subsection{Prototype Control Flow} % (fold)
\label{sub:prototype_control_flow}
\begin{enumerate}
	\item Prototype displays instructions for user to acknowledge.
	\item Keyboard is played with faded prompt.
	\item Current position of carat is underlined.
	\item While prompt is not complete.
	\begin{enumerate}
		\item User moves cursor with directional pad.
		\item User selects character (button for original, right stick for
		new proposed design).
		\item If selected character does not match prompt's, it is underlined
		red and error is registered.
		\item Carat moves to next character.
	\end{enumerate}
	\item User submits text. Return to loop if input does not match prompt.
	\item Task duration is provided and number of errors shown.
\end{enumerate}
% subsection prototype_control_flow (end)
% section high_fidelity_prototype (end)
\end{document}