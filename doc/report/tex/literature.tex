\documentclass[requirements.tex]{subfiles}
\begin{document}
\section{Literature Review} % (fold)
\label{sec:literature_review}

While the general field of text entry has been studied in great detail, text
entry with relation to game controllers specifically is not as refined. However,
there are still plenty of papers that have been written on this topic, each
looking into ways of tackling the same problems that I have found in text entry
with game controllers. The papers being reviewed are those with have some focus
on games controllers, rather than looking at text entry in a more general
setting.

\subsection{Dual Stick Methods} % (fold)
\label{sub:dual_stick_methods}
One common strategy found in these studies regarding controller text entry is
the `dual stick' approach to text entry. These are systems in which the majority
of user input comes from the two analogue sticks. Wilson \& Agrawala
\cite{Wilson:2006:TEU:1124772.1124844} of Microsoft Research propose a system
in which the keyboard is split into two sections, left and right, with the
left stick controlling the left side's cursor, and the right stick acting the
same on the right. The inspiration behind this system was that regular keyboard
use `builds motor memory of the keyboard layout within each hand', since normal
keyboard each sees each hand stick to its own half of the keyboard. The goal of
their proposed system to `exploit such motor memory' in their own system. Their
results showed that their dual stick system was easy to learn albeit with a
slightly increased error rate as well. 

A similar dual stick approach would be found in K\"{o}ltringer et al's
\cite{Koltringer:2007:TWG:1268517.1268536} `TwoStick' paper. TwoStick presents
the unusual approach of providing a 9x9 grid divided into 9 zones (like Sudoku),
with the 9 zones corresponding to the movement of one stick. In each sub-zone,
the other stick controls the character being selected. Their results showed a
predicted outcome of TwoStick's learning curve being a detriment to performance
in initial uses, but as users got used to this new system, they would improve
and type faster than with the traditional QWERTY single selection layout. 
K\"{o}ltringer et al report that there was `only a minute difference' amongst
novice users between their system and the one by Wilson \& Agrawala, which I
found very intriguing.
% subsection dual_stick_methods (end)

\subsection{Stylus Inspired Methods} % (fold)
\label{sub:stylus_inspired_methods}
Another common theme found through researching papers about this topic was that
there had been some approaches that took inspiration from research into text
entry using a stylus, such as for a PDA. The goal in these systems is to
represent character selection through the movement of the joystick, the movement
of the joystick being a representation of the of the character.

EdgeWrite \cite{Wobbrock:2004:JTE:985921.986126}
\cite{Wobbrock:2003:EST:964696.964703} by Wobbrock et al is one such example, in
which movement of the joystick draws a pattern which is recognized as a
character or action. By converting the EdgeWrite technique to joystick, tests
were performed to evaluate performing, finding that the speed of writing had
increased by around one word per minute. However, it was found that this system
was more error prone compared to the selection keyboard (another name for the
standard on-screen keyboard).

Another such stylus based approach study was made by Isokoski \& Raisamo
\cite{Isokoski:2004:QMT:1028014.1028031}, which was based on the Quikwriting
system by Perlin \cite{Perlin:1998:QCS:288392.288613}. This is perhaps the most
unusual of all systems found in the research of this topic, and thus the system
feels very overwhelming at first glance in my opinion. The system works by 
splitting the joystick into 8 sections with one central section too. The
joystick is moved to one of these 8 sections, containing a group of characters,
and is moved again, which selects a specific character 
\cite[Figure~2]{Isokoski:2004:QMT:1028014.1028031}. While originally designed
for stylus, it can be seen why Isokoski \& Raisamo felt it could be used in this
space as well. The results found in this study showed 13 words per minutes
in the joystick space, which is certainly impressive. However, the authors do
concede to the learning difficulty this method presents to the user.
% subsection stylus_inspired_methods (end)

\subsection{Multi-tap} % (fold)
\label{sub:multi-tap}
The final approach that was found in the research of this topic was a multi-tap
approach similar to those found on pre-smartphone mobile phones. A study by
K\"{o}ltringer et al \cite{Koltringer:2007:GCT:1240866.1241033} performed a
comparison between the selection keyboard and a multi-tap system. The multi-tap
system was implemented by Sony for the Playstation 3. However, the results
found showed that  the `alphabetic selection keyboard outperforms the multi-tap
selection keyboard for both novice and expert users', with clear preference
towards the former as well. This shows that multi-tap should not be considered
further.
% subsection multi-tap (end)

\subsection{Conclusions} % (fold)
\label{sub:conclusions}
As Section \ref{sec:market_analysis} will show, the video game industry largely
ignored the studies into this topic. With the exception of Wilson \& Agrawala,
none of these approaches can be found in the current market, which is likely
because it does not deviate far from the current on-screen selection keyboard
found today, merely adding an additional cursor. In addition, the study that
examined an alternative approach already implemented (multi-tap) showed that it
was not a useful text entry system in this problem space, showing the need for
further investigation.

Through all these papers, a common evaluation strategy can be formed. The
metrics that are considered most are the words per minute (WPM) and the error
rates found from using these systems. The studies also allowed users several
hours of experience with the systems through the evaluation process, it is
unlikely that such time will be available for this project, with the ideal goal
being having a two week period to perform my own evaluation.
% subsection conclusions (end)
% section literature_review (end)
\end{document}